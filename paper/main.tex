\documentclass[12pt]{article}
\usepackage[margin=1in]{geometry}
\usepackage{amsmath, amssymb}
\usepackage{graphicx}
\newcommand{\fig}[3]{%
\begin{figure}[h]
\centering
\includegraphics[width=0.85\linewidth]{figures/#1}
\caption{#2}
\label{fig:#3}
\end{figure}}
\usepackage{hyperref}
\usepackage{setspace}
\onehalfspacing

\title{\textbf{A Computational Study of Derivative Pricing under Stochastic Volatility Using the Black--Scholes Framework}}
\author{\textbf{Ekantheswar Bandarupalli}\\
\small Working Paper --- Computational Finance Research\\
\small \texttt{study.ekantheswar@gmail.com}}
\date{\today}

\begin{document}
\maketitle
\thispagestyle{empty}
\vspace{1em}
\noindent\textbf{Abstract. }
The intersection of financial markets and mathematical modeling has long fascinated both practitioners and academics.
While algorithmic trading first sparked my interest in finance, it was the underlying question of how risk and price evolve
that drew me toward stochastic calculus and derivative pricing. This study presents a computational approach to pricing European options
using the Black--Scholes model and compares theoretical values with real S\&P 500 (SPX) market data. We highlight model fit, limitations,
and directions toward stochastic volatility models.

\newpage
\setcounter{page}{1}
\section{Introduction}
My initial interest in financial markets began through trading, where market behavior seemed both systematic and unpredictable.
What began as an exploration of trading strategies evolved into a deeper curiosity about the mathematics governing price movements,
volatility, and risk. I soon realized that true insight lies not in execution alone, but in the stochastic models that shape financial dynamics.
The objective of this work is to bridge theoretical finance with computational implementation.
By applying the Black--Scholes model to real SPX option data, this paper evaluates accuracy, limitations, and potential extensions
toward stochastic volatility (e.g., Heston).

\section{Methodology}
We consider the standard Black--Scholes setting with underlying price $(S_t)_{t\ge 0}$ following geometric Brownian motion under the risk--neutral measure:
\begin{equation}
dS_t = r S_t\,dt + \sigma S_t\, dW_t,
\end{equation}
where $r$ is the constant risk--free rate, $\sigma$ the (assumed constant) volatility, and $(W_t)$ a standard Brownian motion.
For a European call with strike $K$ and maturity $T$, the no--arbitrage price at $t=0$ is
\begin{equation}
C = S_0 \Phi(d_1) - K e^{-rT} \Phi(d_2), \quad
d_{1,2} = \frac{\ln(S_0/K) + (r \pm \tfrac{1}{2}\sigma^2)T}{\sigma\sqrt{T}},
\end{equation}
and for a put
\begin{equation}
P = K e^{-rT} \Phi(-d_2) - S_0 \Phi(-d_1),
\end{equation}
where $\Phi$ denotes the standard normal CDF.

We compute the \emph{Greeks} to assess sensitivities: Delta ($\partial C/\partial S$), Gamma ($\partial^2 C/\partial S^2$),
Vega ($\partial C/\partial \sigma$), Theta ($\partial C/\partial T$), and Rho ($\partial C/\partial r$), using closed--form expressions.
Empirically, we retrieve SPX option chains via public data and form the time--to--maturity $T$ and strike grids around the at--the--money region.
We compare market mid quotes with the Black--Scholes price computed using market implied volatilities and report pricing errors by strike.

\paragraph{Limitations.} The constant volatility assumption leads to systematic mispricing patterns (e.g., volatility smile/smirk).
In subsequent work we outline a stochastic volatility extension (Heston) where $d\nu_t = \kappa(\theta-\nu_t)dt + \xi \sqrt{\nu_t}\, dZ_t$
and price via Fourier methods or simulation.

\section{Results}
We analyze the nearest--maturity SPX option chain and compare market mid prices with Black--Scholes values computed using market implied volatilities. Figure~\ref{fig:pricing-error} shows the pricing error (market price minus Black--Scholes price) across strikes near the at--the--money (ATM) region.

\fig{pricing_error_vs_strike.png}{Market vs.\ Black--Scholes mispricing by strike for SPX calls near the ATM region and nearest expiry. Positive values indicate the market is priced above the baseline model.}{pricing-error}

The error profile displays systematic curvature across strikes, consistent with well--known \emph{volatility smile/smirk} effects that violate the constant--volatility assumption. In particular, far OTM/ITM regions tend to exhibit larger absolute deviations, while ATM errors compress as maturity approaches.\footnote{Exact patterns depend on the date, maturity, and liquidity of the option chain; figures are regenerated programmatically from the accompanying notebook.}

We also report illustrative Greek curves around ATM using the same maturity and an IV representative of the observed chain (median IV). Delta transitions sharply through $0.5$ at ATM, while Gamma peaks near ATM, reflecting higher sensitivity of Delta to price changes:

\fig{delta_vs_strike.png}{Delta as a function of strike (nearest maturity, representative IV). The transition around ATM reflects the probability mass under the risk--neutral distribution.}{delta}
\fig{gamma_vs_strike.png}{Gamma as a function of strike (nearest maturity, representative IV). The peak at ATM indicates heightened curvature (sensitivity of Delta) for short maturities.}{gamma}

These diagnostics validate the computational pipeline and highlight model limitations that motivate richer dynamics (e.g., stochastic volatility).

\paragraph{Discussion.}
Under Black--Scholes, a single constant $\sigma$ must reconcile prices across strikes and maturities. Empirically, equity index options exhibit strike--dependent implied volatilities (smile/smirk) and maturity structures (term structure), producing systematic mispricing when the constant--volatility assumption is imposed. These findings motivate extensions such as local volatility or stochastic volatility. In forthcoming work we will implement Heston dynamics and compare fit and stability to the Black--Scholes baseline.


\section{Conclusion}
Black--Scholes provides a transparent baseline and a reproducible pipeline from data to price and sensitivities.
Observed deviations highlight the need for richer dynamics; future work will implement and compare a Heston specification.

\bibliographystyle{plain}
\bibliography{references}
\end{document}
